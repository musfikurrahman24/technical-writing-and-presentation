\documentclass[conference]{IEEEtran}
\IEEEoverridecommandlockouts
% The preceding line is only needed to identify funding in the first footnote. If that is unneeded, please comment it out.
\usepackage{cite}
\usepackage{amsmath,amssymb,amsfonts}
\usepackage{algorithmic}
\usepackage{graphicx}
\usepackage{textcomp}
\usepackage{xcolor}
\def\BibTeX{{\rm B\kern-.05em{\sc i\kern-.025em b}\kern-.08em
    T\kern-.1667em\lower.7ex\hbox{E}\kern-.125emX}}
\begin{document}

\title{|Coin-Based Mobile Charging System Using Arduino*\\
{\footnotesize \textsuperscript{}}
\thanks{}
}

\author{\IEEEauthorblockN{S M Musfikur Rahman \\ Student No.: 190224} 
\IEEEauthorblockA{\textit{Computer Science and Engineering Discipline} \\
\textit{Khulna University}\\
Khulna, Bangladesh \\
190224@ku.ac.bd}
\and
\IEEEauthorblockN{Utsa Debnath \\ Student No.: 200242}
\IEEEauthorblockA{\textit{Computer Science and Engineering Discipline}\\
\textit{Khulna University}\\
Khulna, Bangladesh \\
200242@ku.ac.bd}
}

\maketitle

\section{RELATED WORK}
There have been several works in the past that have explored the concept of a coin-based mobile charging system using Arduino. Here are a few related works:

\subsection{Coin-Based Universal Mobile Battery Charger}\label{AA11}
The coin-based mobile battery charger developed in this paper is providing a unique service to the rural public where grid power is not available for partial/full daytime and a source of revenue for site providers. The coin-based mobile battery charger can be quickly and easily installed outside any business
premises.In this paper, a novel method of charging mobile batteries of different manufacturers using solar power has been designed and developed.

\subsection{Coin-Based Mobile Charger}\label{AA}

The goal of this idea is to insert a coin utilizing a mobile phone charging in public places. The IR receiver is utilized on the receiver side to receive the IR signal. To adjust the polarity of the pulse in SCU input, place a coin between the IR transmitter and receiver. The SCU converts low pulses to high pulses, which are then inverted in the inverter. The 555IC is used as a timer to generate a high pulse for a set amount of time. The SCU's gain is utilized to convert low pulses to high pulses, and the output is sent into the driver circuit's input. The driver circuit ensures that the relay's input voltage is sufficient. The coin acceptor determines if the coin is valid or not after it has been entered. The power is only available for a limited time for each unit of pricing. Based on the number of coins inserted, the Arduino can calculate the time. This approach is beneficial to persons who travel vast distances.


\subsection{Secured Coin-Based Cell Phone Charger with RFID}
In this work, the authors proposed a secure and efficient coin-based mobile charging system that employed an RFID system for coin authentication and a power management system for efficient power utilization.

\subsection{MOBILE CHARGING VENDING MACHINE}
This work proposes a coin-operated power bank that can be used to charge mobile phones. The system employs an LCD display to show the status of the charging process and the amount of money inserted.

These works demonstrate that the concept of a coin-based mobile charging system using Arduino has been well-researched, and various solutions have been proposed to address the challenges posed by this system.



\begin{thebibliography}{00}
\bibitem{b11}D. M. J. Sadiq, “Coin Based Mobile Charger.” International Journal for Research in Applied Science and Engineering Technology, vol. 10, no. 8, pp. 1241-1251, 2022, doi: 10.22214/ijraset.2022.44625.
\bibitem{b1}A. ., “COIN BASED MOBILE CHARGER USING RFID WITH PV FOR PUBLIC USAGE.” International Journal of Research in Engineering and Technology, vol. 5, no. 4, pp. 32-35, 2016, doi: 10.15623/ijret.2016.0504007.
\bibitem{b3} M. Varadarajan, “Coin Based Universal Mobile Battery Charger.” IOSR Journal of Engineering, vol. 2, no. 6, pp. 1433-1438, 2012, doi: 10.9790/3021-026114331438.

\end{thebibliography}


\end{document}
